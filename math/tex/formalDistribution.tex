\M
Let $U$ be a vector space over $\CC$. A series of the form
\begin{equation}
\sum_{m,n\in\ZZ}a_{mn}z^{m}\omega^{n}
\end{equation}
where $a_{mn}\in U$ we call a \define{Formal Distribution} in the
indeterminates $z,\omega$ taking values in $U$. They form a vector space
denoted $U[[z,z^{-1},\omega,\omega^{-1},\dots]]$. 

\M
Given a formal distribution (or more weakly, a formal Laurent polynomial)
$a(z) = \sum a_{n}z^{n}$ we define its \define{Residue} by the usual
formula
\begin{equation}
\Res[z]{a(z)} = a_{-1}.
\end{equation}
Since
\begin{equation}
\Res[z]{\partial a(z)}=0
\end{equation}
we have the usual integration by parts formula
\begin{equation}
\Res[z]{\partial a(z)b(z)}=-\Res[z]{a(z)\partial b(z)}
\end{equation}
where $\partial a(z)=\sum na_{n}z^{n-1}$.

\M
Let
\begin{equation}
\CC(z)=\CC[z,z^{-1}]
\end{equation}
bt the algebra of Laurent polynomials in $z$. We have a nondegenerate
pairing
\begin{equation}
\<-,-\>\colon U[[z,z^{-1}]]\times\CC[z,z^{-1}]\to U
\end{equation}
defined by
\begin{equation}
\<f,\varphi\>=\Res[z]{f(z)\varphi(z)}
\end{equation}
Hence Laurent polynomials should be seen as ``test functions'' for the
formal distributions.

\M
We introduce the most famous of distributions: the Dirac Delta
function. We define it as
\begin{equation}
\delta(z-\omega)
=z^{-1}\sum_{n\in\ZZ}(z/\omega)^{n}
\end{equation}
\begin{rmk}
Kac has some confusing reasoning for $\partial^{k}\delta(z-\omega)$ that
I don't quite follow. But using our given definition, it makes more sense.
\end{rmk}

\begin{xca}
Prove $\delta(z-\omega)=\delta(\omega-z)$.
\end{xca}
\begin{xca}
What is $\Res[z]{(z-\omega)^{n}}$?
\end{xca}
\begin{xca}
Prove $\partial^{k}_{z}\delta(z-\omega)=(-\partial_{\omega})^{k}\delta(z-\omega)$.
\end{xca}
\begin{xca}\label{xca:deltaDerivative}%
Prove $(z-\omega)\partial^{k+1}_{\omega}\delta(z-\omega)=\partial^{k}_{\omega}\delta(z-\omega)$.
\end{xca}
\begin{xca}\label{xca:usefulForCor3}%
Prove $(z-\omega)^{k+1}\partial^{k}_{\omega}\delta(z-\omega)=0$.
\end{xca}

\N{Proposition}\label{prop:integrateDeltaFn}
Let $f(z)\in U[[z,z^{-1}]]$ be any formal distribution, then
\begin{equation}
\Res[z]{f(z)\delta(z-\omega)}=f(\omega).
\end{equation}


\M
When does a formal distribution
\begin{equation}
a(z,\omega)=\sum_{m,n\in\ZZ}a_{mn}z^{m}\omega^{n}
\end{equation}
has an expansion of the form
\begin{equation}
a(z,\omega)=\sum^{\infty}_{j=0}c^{j}(\omega)\partial^{j}_{\omega}\delta(z-\omega)?
\end{equation}
Well, multiplying both sides by $(z-\omega)^{n}$ and taking the reside,
we find
\begin{equation}
c^{n}(\omega)=\Res[z]{a(z,\omega)\cdot(z-\omega)^{n}}
\end{equation}
Can we do this for any formal distribution? Not really. But we can
construct a transformation $\pi$ such that
\begin{equation}
\pi
a(z,\omega)=\sum^{\infty}_{j=0}\bigl[\Res[z]{a(z,\omega)\cdot(z-\omega)^{j}}\bigr]\partial^{j}_{\omega}(z-\omega).
\end{equation}
What is this transformation $\pi$?

\begin{prop}
The transformation $\pi$ is a projection (i.e., $\pi^{2}=\pi$). Moreover
\begin{equation}
\ker(\pi)=\{a(z,\omega)| a(z,\omega)=\sum_{m\in\NN,n\in\ZZ}a_{mn}z^{m}\omega^{n}\}.
\end{equation}
\end{prop}

\begin{cor}
Any formal distribution $a(z,\omega)$ may be represented uniquely as
\begin{equation}
a(z,\omega)=b(z,\omega) + \sum^{\infty}_{j=0}c^{j}(\omega)\partial^{j}_{\omega}\delta(z-\omega)
\end{equation}
where $b(z,\omega)\in\ker(\pi)$.
\end{cor}
\begin{cor}\label{cor:kerMultOp}
Let $\mu_{N}$ be the multiplication operator by $(z-\omega)^{N}$. THen
its nullspace in $U[[z,z^{-1},\omega,\omega^{-1}]]$ is
\begin{equation}
\ker(\mu_{N})=\sum^{N-1}_{j=0}\partial^{j}_{\omega}\delta(z-\omega)U[[\omega,\omega^{-1}]].
\end{equation}
Any element $a(z,\omega)\in\ker(\mu_N)$ is uniquely written of the form
\begin{equation}
a(z,\omega)=\sum^{N-1}_{j=0}c^{j}(\omega)\partial^{j}_{\omega}\delta(z-\omega)
\end{equation}
where $c^{j}(\omega)$ is the ``usual coefficients''.
\end{cor}
\begin{proof}
By exercise \ref{xca:usefulForCor3}, we see
\begin{equation}
(z-\omega)^{j+1}\partial_{\omega}^{j}\delta(z-\omega)=0,
\end{equation}
proving $\partial_{\omega}^{j}\delta(z-\omega)$ lives in the
nullspace. But observe multipying through by $c^{j}(\omega)$ changes
\emph{nothing}. So $\partial_{\omega}^{j}\delta(z-\omega)$ acts like a
basis for $\ker(\mu_N)$.
\end{proof}
\begin{rmk}
The coefficients $c^{j}(\omega)$ are like Fourier coefficients.
\end{rmk}
\N*{Notation}
Often formal distributions will be written as 
\begin{equation}
a(z)=\sum_{n\in\ZZ}a_{(n)}z^{-n-1}
\end{equation}
where
\begin{equation}
a_{(n)}=\Res[z]{a(z)z^{n}}.
\end{equation}
Using this notation, the expansion in Corollary \ref{cor:kerMultOp} is
equivalent to 
\begin{equation}
a_{(m,j)}=\sum^{N-1}_{j=0}\binom{m}{j}c^{j}_{(m+n-j)}.
\end{equation}

\subsection{Locality}
\M
We are interested when $U$ is an associative superalgebra (i.e., a
$\ZZ_2$-graded associative algebra). This happens when $V$ is a
superlinear space, and we have
\begin{equation}
U=\End(V)
\end{equation}
We denote the superbracket as
\begin{equation}
\superb{a}{b}=ab-(-1)^{\alpha\beta}ba.
\end{equation}
where $\alpha=\deg(a)$, $\beta=\deg(b)$.

\N{Definition}
Two formal distributions $a(z)$, $b(z)$ with values in an associative
superalgebra $U$ are called \define{Mutually Local} if in
$U[[z,z^{-1},\omega,\omega^{-1}]]$ one has
\begin{equation}
(z-\omega)^N\superb{a(z)}{b(z)}=(z-\omega)^N\superb{b(z)}{a(z)}=0
\end{equation}
for $N\gg0$.

NB: the parity of a formal distribution is the parity of all its
coefficients.

\begin{xca}
Prove or find a counter-example: if $a(z)$, $b(\omega)$ are mutually
local, then $\partial_{z}a(z)$ and $b(\omega)$ are mutually local.
\end{xca}

\M
We introduce notation.
Let
\begin{equation}
a(z)=\sum_{n\in\ZZ}a_{(n)}z^{-n-1},
\end{equation}
then we denote
\begin{equation}
a(z)_{-}=\sum_{n\geq0}a_{(n)}z^{-n-1},
\quad\mbox{and}\quad
a(z)_{+}=\sum_{n<0}a_{(n)}z^{-n-1}.
\end{equation}
This is not the only way to decompose $a(z)$ into ``positive'' and
``negative'' parts, but it is the \emph{only} way which also satisfies
\begin{equation}
\bigl(\partial a(z)\bigr)_{\pm}=\partial\bigl(a(z)_{\pm}\bigr)
\end{equation}

Given formal distributions $a(z)$, $b(\omega)$, we define
\begin{equation}
\normOrd{a(z)b(\omega)} = a(z)_{+}b(\omega)+(-1)^{\alpha\beta}b(\omega)a(z)_{-},
\end{equation}
where $\alpha=\deg(a)$, $\beta=\deg(b)$. We borrow terminology from
physicists, and call this the \define{Normal Ordering} of $a(z)b(\omega)$.

\begin{xca}
Is it true that $\normOrd{a(z)b(\omega)}=\normOrd{b(\omega)a(z)}$?
\end{xca}
\begin{xca}
Prove
$a(z)b(\omega)=\superb{a(z)_{-}}{b(\omega)}+\normOrd{a(z)b(\omega)}$.
\end{xca}
\begin{xca}
Prove
$(-1)^{\alpha\beta}b(\omega)a(z)=-\superb{a(z)_{+}}{b(\omega)}+\normOrd{a(z)b(\omega)}$.
\end{xca}
\begin{xca}
Find $\Res[z]{\normOrd{a(z)b(\omega)}}$ and 
$\Res[\omega]{\normOrd{a(z)b(\omega)}}$.
\end{xca}

\N{Theorem}
Let $U$ be an associative super algebra.
The following properties are equivalent
\begin{enumerate}
\item
  $\displaystyle\superb{a(z)}{b(\omega)}=\sum^{N-1}_{j=0}c^{j}(\omega)\partial_{\omega}^{j}\delta(z-\omega)$
  where $c^{j}(\omega)\in U[[\omega,\omega^{-1}]]$.
\item
\begin{align*}
\superb{a(z)_{-}}{b(\omega)} &= \sum^{N-1}_{j=0}\bigl(i_{z\omega}(z-\omega)^{-j-1}\bigr)c^{j}(\omega)\\
\superb{a(z)_{+}}{b(\omega)} &= \sum^{N-1}_{j=0}\bigl(i_{z\omega}(z-\omega)^{-j-1}\bigr)c^{j}(\omega)
\end{align*}
where $i_{z\omega}(z-\omega)^{j+1}=\sum_{m}z^{-m-1}\binom{m}{j}\omega^{m-j}$ and
$\partial_{\omega}^{j}\delta(z-\omega)-i_{z\omega}(z-\omega)^{-j-1}=i_{\omega
z}(z-\omega)^{-j-1}$.
\item $\displaystyle
a(z)b(\omega)=\left[\sum^{N-1}_{j=0}i_{z\omega}(z-\omega)^{-j-1}c^{j}(\omega)\right]+\normOrd{a(z)b(\omega)}$,
and
$\displaystyle
(-1)^{\alpha\beta}b(\omega)a(z)=\left[\sum^{N-1}_{j=0}c^{j}(\omega)i_{\omega z}
(z-\omega)^{-j-1}\right]+\normOrd{a(z)b(\omega)}$.
\item $\displaystyle\superb{a_{(m)}}{b_{(n)}}=\sum^{N-1}_{j=0}\binom{m}{j}c^{j}_{(m+n-j)}$
\item $\displaystyle\superb{a_{(m)}}{b(\omega)}=\sum_{j=0}\binom{m}{j}c^{j}(\omega)\omega^{m-j}$
\item
$\displaystyle\superb{a_{(m)}}{b_{(n)}}=\sum_{j=0}p_{j}(m)d^{j}_{m+n}$
where $p_{j}(x)$ are some polynomial, $d^{j}_{m+n}\in U$.
\item $a(z)b(\omega)=\left[i_{z\omega}(z-\omega)^{-N}\right]c(z,\omega)$
and $(-1)^{\alpha\beta}b(\omega)a(z)=[i_{\omega z}(z-\omega)^{-N}]c(z,\omega)$
for some formal distribution $c(z,\omega)$.
\end{enumerate}

\M
Physicists abuse notation writing
\begin{equation}
\superb{a(z)}{b(\omega)}=\sum^{N-1}_{j=0}c^{j}(\omega)\partial_{\omega}^{j}\delta(z-\omega)
\end{equation}
instead of
\begin{equation}
\superb{a(z)}{b(\omega)}=\sum^{N-1}_{j=0}\frac{c^{j}(\omega)}{(z-\omega)^{j+1}}+\normOrd{a(z)b(\omega)}
\end{equation}
or often just the singular part
\begin{equation}
\superb{a(z)}{b(\omega)}\sim\sum^{N-1}_{j=0}\frac{c^{j}(\omega)}{(z-\omega)^{j+1}}.
\end{equation}
Either case is called the \define{Operator Product Expansion}\index{Operator Product Expansion}. The
previous theorem tells us the singular part of the operator product
expansion encodes the brackets between all the coefficients of mutually
local formal distributions $a(z)$ and $b(\omega)$.

\M
We introduce (for each $n\in\NN$) the $n^{th}$ product
$a(\omega)_{(n)}b(\omega)$ on the space of formal distributions by the
formula
\begin{equation}
a(\omega)_{(n)}b(\omega)=\Res[z]{\superb{a(z)}{b(\omega)}(z-\omega)^{n}}
\end{equation}
Observe the operator product expansion becomes
\begin{equation}
a(z)b(\omega)=\sum^{N-1}_{j=0}\frac{a(\omega)_{(j)}b(\omega)}{(z-\omega)^{j+1}}+\normOrd{a(z)b(\omega)}.
\end{equation}

\N{Proposition} 
(a) For any two formal distributions $a(\omega)$ and $b(\omega)$, and
for any $n\in\NN$, we have
\begin{equation}
\partial a(\omega)_{(n)}b(\omega)=-na(\omega)_{(n-1)}b(\omega).
\end{equation}
Moreover $\partial$ is a derivation of all $n^{th}$ products.

(b) For any mutually local formal distributions $a(\omega)$ and
$b(\omega)$, and for any $n\in\NN$, we have
\begin{equation}
a(\omega)_{(n)}b(\omega)=(-1)^{\alpha\beta}\sum^{\infty}_{j=0}(-1)^{j+n+1}\partial^{j}\bigl(b(\omega)_{(n+j)}a(\omega)\bigr).
\end{equation}

(c) For any three formal distributions $a(\omega)$, $b(\omega)$ and
$c(\omega)$, and for any $m,n\in\NN$, we have
\begin{equation}
\begin{split}
a(\omega)_{(m)}\bigl(b(\omega)_{(n)}c(\omega)\bigr)
&=\sum^{m}_{j=0}\binom{m}{j}\bigl(a(\omega)_{(j)}b(\omega)\bigr)_{(m+n-j)}c(\omega)\\
&\quad+(-1)^{\alpha\beta}b(\omega)_{(n)}\bigl(a(\omega)_{(m)}c(\omega)\bigr).
\end{split}
\end{equation}

\begin{proof}
(a)
We want to prove 
\begin{equation}
\partial a(\omega)_{(n)}b(\omega)=-na(\omega)_{(n)}\partial b(\omega).
\end{equation}
We see that the left hand side is equal to
\begin{subequations}
\begin{align}
\partial a(\omega)_{(n)}b(\omega)
&=\partial_{\omega}\Res[z]{\superb{a(z)}{b(\omega)}\cdot(z-\omega)^{n}}\\
&=\Res[z]{\superb{a(z)}{\partial b(\omega)}\cdot(z-\omega)^{n}}
-n\Res[z]{\superb{a(z)}{b(\omega)}\cdot(z-\omega)^{n-1}}\\
&=a(\omega)_{(n)}\partial b(\omega)-na(\omega)_{(n-1)}b(\omega)\\
&=a(\omega)_{(n)}\partial b(\omega)+\bigl(\partial
a(\omega)_{(n)}\bigr)b(\omega)
\end{align}
\end{subequations}
where the last step is just Leibniz's rule.

(b) We have
\begin{equation}
a(\omega)_{(n)}b(\omega)
= \Res[z]{\superb{a(z)}{b(\omega)}(z-\omega)^{n}}
\end{equation}
using the definition of the $n^{th}$ product. Using the definition of
the super bracket we have
\begin{equation}
\Res[z]{\superb{a(z)}{b(\omega)}(z-\omega)^{n}}=-(-1)^{\alpha\beta}\Res[z]{\superb{a(\omega)}{a(z)}\cdot(z-\omega)^{n}}.
\end{equation}
Using corollary \ref{cor:kerMultOp} we have
\begin{equation}
\Res[z]{\superb{a(\omega)}{a(z)}\cdot(z-\omega)^{n}}
=\Res[z]{\sum^{\infty}_{j=1}(-1)^{j}\partial_{\omega}^{j}\delta(z-\omega)b(z)_{(j)}a(z)(z-\omega)^{n-1}}.
\end{equation}
Exercise \ref{xca:deltaDerivative} and the definition of the $n^{th}$
product gives us
\begin{subequations}
\begin{align}
a(\omega)_{(n)}b(\omega)&=(-1)^{\alpha\beta}\Res[z]{\sum^{\infty}_{j=1}(-1)^{j+1}\partial_{\omega}^{j+1}\delta(z-\omega)b(z)_{(j)}a(z)}\\
&=(-1)^{\alpha\beta}\Res[z]{\sum^{\infty}_{j=1}(-1)^{n+1}\partial_{z}^{j}\delta(z-\omega)b(z)_{(j+n)}a(z)}
\end{align}
\end{subequations}
We integrate by parts and use proposition \ref{prop:integrateDeltaFn} to
find
\begin{subequations}
\begin{align}
a(\omega)_{(n)}b(\omega)
&=(-1)^{\alpha\beta}\Res[z]{\sum^{\infty}_{j=1}(-1)^{j+1+n}\delta(z-\omega)\partial_{z}^{j}\bigl(b(z)_{(j+n)}a(z)\bigr)}\\
&=(-1)^{\alpha\beta}\sum^{\infty}_{j=1}(-1)^{j+1+n}\partial^{j}\bigl(b(\omega)_{(j+n)}a(\omega)\bigr)
\end{align}
\end{subequations}
which is what we wanted!

(c) We recall the definition of the $n^{th}$ product, which tells us
\begin{subequations}
\begin{equation}
a(\omega)_{(m)}\bigl(b(\omega)_{(n)}c(\omega)\bigr)
=a(\omega)_{(m)}\Res[u]{\superb{b(u)}{c(\omega)}\cdot(u-\omega)^{n}}
\end{equation}
and applying it again
\begin{equation}
a(\omega)_{(m)}\bigl(b(\omega)_{(n)}c(\omega)\bigr)
=\Res[z]{\Res[u]{\superb{a(z)}{\superb{b(u)}{c(\omega)}}(u-\omega)^{n}(z-\omega)^{m}}}.
\end{equation}
\end{subequations}
Using the super-Jacobi super identity gives us
\begin{multline}\label{eq:forPfOfPtCofWhackyProp}
\Res[z]{\Res[u]{\superb{a(z)}{\superb{b(u)}{c(\omega)}}(u-\omega)^{n}(z-\omega)^{m}}}\\
=\Res[z]{\Res[u]{\superb{\superb{a(z)}{b(u)}}{c(\omega)}(u-\omega)^{n}(z-\omega)^{m}}}\\
+(-1)^{\alpha\beta}\Res[z]{\Res[u]{\superb{b(u)}{\superb{a(z)}{c(\omega)}}(u-\omega)^{n}(z-\omega)^{m}}}
\end{multline}
Observe
\begin{equation}
\Res[z]{\Res[u]{
}} = b(\omega)_{(n)}\bigl(a(\omega)_{(m)}c(\omega)\bigr)
\end{equation}
Combinatorics tells us
\begin{equation}
(z-\omega)^{m}(u-\omega)^{n}=\sum^{m}_{j=0}\binom{m}{j}(z-u)^{j}(u-\omega)^{m-j+n}.
\end{equation}
When we substitute this back into Eq \eqref{eq:forPfOfPtCofWhackyProp}
and this proves (c).
\end{proof}

\begin{cor}
(a) If $a(z)$ and $b(z)$ are formal distributions, then
$\superb{a_{(n)}}{b(z)}=0$ if and only if $a(z)_{(n)}b(z)=0$.

(b) If $a(z)$ is an odd formal distribution, then $a_{(n)}^{2}=0$ if and
only if $\Res[z]{a(z)_{(n)}a(z)}=0$.

(c) Let $\mathcal{A}$ be a space consisting of mutually local formal
$U$-valued distributions in $\omega$ which is $\partial$-invariant and
closed with respect to the $0^{th}$ product. Then with respect to the
$0^{th}$ product, $\partial\mathcal{A}$ is a 2-sided ideal of
$\mathcal{A}$, and $\mathcal{A}/\partial\mathcal{A}$ is a Lie
superalgebra. Moreover the $0^{th}$ product defines on $\mathcal{A}$ the
structure of a left $(\mathcal{A}/\partial\mathcal{A})$-module.
\end{cor}

\subsection{Taylor's Formula}

\M
Taylor's formula helps compute operator product expansions. We have some
notation first: given some formal distribution 
\begin{equation}
a(z)=\sum_{n}a_{n}z^{n}
\end{equation}
we may construct a formal distribution in $z$ and $\omega$:
\begin{equation}
i_{z,\omega}a(z-\omega)\eqdef\sum_{n} a_{n}i_{z,\omega}(z-\omega)^{n}.
\end{equation}
Simplifying notation further, we consider the formal distribution
$a(z-\omega)$ in $z$ and $\omega$ in the domain $|z|>|\omega|$.

\begin{prop}[Taylor's Formula]
Let $a(z)$ be a formal distriubtion. Then we have the following equality
of formal distributions in $z$ and $\omega$ in the domain $|z|>|\omega|$
\begin{equation}
a(z+\omega)=\sum^{\infty}_{j=0}\partial_{z}^{j}a(z)\omega^{j}.
\end{equation}
\end{prop}
\begin{proof}
We have $a(z)=\sum a_{n}z^{n}$, then
\begin{equation}
\partial_{z}^{j}a(z)=\sum_{n}\binom{n}{j}a_{n}z^{n-j}.
\end{equation}
We plug $(z+\omega)$ into $a(-)$, first recalling the Binomial theorem states
\begin{equation}
(z+\omega)^{n}=\sum^{\infty}_{j=0}\binom{n}{j}z^{n-j}\omega^{j}
\end{equation}
thus we obtain the desired result.
\end{proof}

\begin{rmk}
We see that $a(\omega)=\sum\partial^{j}a(z)(\omega-z)^{j}$, which proves
another result (for free)!
\end{rmk}

\begin{thm}
Let $a(z)$ be a formal distribution and $N$ be a non-negative
integer. Then the following equality of formal distributions in $z$ and
$\omega$ 
\begin{equation}\label{Eq:thm:formalDistributionEquality}
\partial_{\omega}^{N}\delta(z-\omega)a(z)=\partial_{\omega}^{N}\delta(z-\omega)\sum^{N}_{j=0}\partial^{j}a(\omega)(z-\omega)^{j}.
\end{equation}
\end{thm}
\begin{proof}
To prove this, we need to see how each side of
Eq \eqref{Eq:thm:formalDistributionEquality} acts on an arbitrary
Laurent polynomial $f(z)$.
We see
\begin{equation}
\Res[z]{\partial_{z}^{N}\delta(z-\omega)a(z)f(z)}=\sum\partial^{j}a(\omega)\Res[z]{\partial_{z}^{N}\delta(z-\omega)\cdot(z-\omega)^{j}f(z)};
\end{equation}
integrate by parts $N$ times gives us
\begin{equation}
\Res[z]{\delta(z-\omega)\partial^{N}\bigl(a(z)f(z)\bigr)}
=\sum\partial^{j}a(\omega)\Res[z]{\delta(z-\omega)\partial_{z}^{N}\bigl((z-\omega)^{j}f(z)\bigr)}
\end{equation}
which, using the Leibniz rule, is
\begin{equation*}
\partial^{N}\bigl(a(z)f(z)\bigr)=\sum\partial^{j}a(\omega)\binom{N}{j}\partial^{N-j}f(\omega).\qedhere
\end{equation*}
\end{proof}

\subsection{Current Algebras}
\M
Consider the oscillator algebra $\mathfrak{s}$. This is a Lie algebra
with basis $\alpha_n$ ($n\in\ZZ$), $K$ with commutation relations
\begin{equation}
[\alpha_m,\alpha_n]=m\delta_{m,-n}K,\quad\mbox{and}\quad
[K,\alpha_m]=0.
\end{equation}
Let $U=U(\mathfrak{s})$ be its universal enveloping algebra.

\M
Consider the formal distribution in $U$
\begin{equation}
\alpha(z)=\sum_{n\in\ZZ}\alpha_nz^{-n-1}.
\end{equation}
We find
\begin{equation}
[\alpha(z),\alpha(\omega)]=\partial_{\omega}\delta(z-\omega)K.
\end{equation}
This means $\alpha$ is local with respect to itself, with the operator
product expansion
\begin{equation}
\alpha(z)\alpha(\omega)\sim\frac{K}{(z-\omega)^{2}}.
\end{equation}
The even formal distribution is called a \define{Free Boson}.

\M
Let $\mathscr{G}$ be a Lie superalgebra equipped with an invariant
supersymmetric bilinear form $(-|-)$. ``Invariant'' means
\begin{equation}
(\superb{a}{b}|c)=(a|\superb{b}{c})
\end{equation}
for all $a,b,c\in\mathscr{G}$, and ``supersymmetric'' means
\begin{equation}
(a|b)=(-1)^{\alpha\beta}(b|a).
\end{equation}
The \define{Affinization} of $\mathscr{G}$ is the Lie superalgebra
\begin{equation}
\widehat{\mathscr{G}}=\CC[t,t^{-1}]\otimes_{\CC}\mathscr{G}+\CC[K]
\end{equation}
with a $\ZZ/2\ZZ$-grading extending that of $\mathscr{G}$ by
\begin{equation}
p(t)=p(K)=0
\end{equation}
and commutation relations
\begin{equation}
\begin{split}
\superb{a_m}{b_n} &= \superb{a}{b}_{m+n}+m(a|b)\delta_{m,-n}K\\
\superb{K}{\widehat{\mathscr{G}}}&=0.
\end{split}
\end{equation}
We abbreviate $at^{m}=a\otimes t^{m}$ as $a_{m}$. Also note that
$\mathscr{G}\otimes\CC[t,t^{-1}]$ is the loop algebra, and
$\widehat{\mathscr{G}}$ its central extension.

\M
We introduce formal distributions with values in
$U(\widehat{\mathscr{G}})$ which are \define{Currents}
\begin{equation}
a(z)=\sum_{n\in\ZZ}a_{n}z^{-n-1}
\end{equation}
for $a_{*}\in\mathscr{G}$. Then we find
\begin{equation}
\superb{a(z)}{b(\omega)} = 
\delta(z-\omega)\superb{a}{b}(\omega) + \partial_{\omega}\delta(z-\omega)\cdot(a|b)\cdot{K}
\end{equation}
hence all the currents are mutually local with respect to the operator
product expansion:
\begin{equation}
a(z)b(\omega)\sim\frac{\superb{a}{b}(\omega)}{z-\omega}+\frac{(a|b)K}{(z-\omega)^{2}}
\end{equation}

\M
We generalize this process (or ``super-size'' it), obtaining
a \define{Superaffinization}, which is a central extension of the super
loop super-algebra:
\begin{equation}
\widehat{\mathscr{G}}_{\text{super}}=\CC[t,t^{-1},\theta]\otimes_{\CC}\mathscr{G}+\CC[K]
\end{equation}
where $\theta^2=0$ and $p(\theta)=1$, and the remaining operator product
expansions are as follows: For $a\in\mathscr{G}$ we define
the \define{Super-Current}
\begin{equation}
\bar{a}(z)=\sum_{n\in\ZZ}a_{n+1/2}z^{-n-1}
\end{equation}
where $a_{n+1/2}=t^{n}\theta\otimes a$. The super currents $\bar{a}(z)$
are pair-wise local and also local with respect to the currents, and the
remaining operator product expansions are
\begin{equation}
a(z)\bar{b}(\omega)\sim\frac{\overline{\superb{a}{b}}(\omega)}{z-\omega},
\quad\mbox{and}\quad
\bar{a}(z)\bar{b}(\omega)\sim\frac{(a|b)K}{z-\omega}.
\end{equation}
They form a subalgebra.

\M
Lets consider a different construction. Let $A$ be a superspace with a
skew-supersymmetric bilinear form, i.e., 
\begin{equation}
(\varphi|\psi)=(-1)^{p(\varphi)}(\psi|\varphi)
\end{equation}
which implies $(\mbox{odd}|\mbox{even})=0$. The \define{Clifford
Affinization} of $\bigl(A,(-|-)\bigr)$ is the Lie superalgebra
\begin{equation}
C_{A}=\CC[t,t^{-1}]\otimes_{\CC}A + \CC[K]
\end{equation}
with the commutation relations (letting $m,n\in\ZZ+1/2$ and
$\varphi,\psi\in A$)
\begin{equation}
\superb{\varphi_{m}}{\psi_{n}}=(\varphi|\psi)\delta_{m,-n}K
\end{equation}
and
\begin{equation}
\superb{C_{A}}{K}=0
\end{equation}
where $\varphi_{m}=t^{m-1/2}\otimes\varphi$. The formal distribution 
\begin{equation}
\varphi(z)=\sum_{n\in\ZZ}\varphi_{n+1/2}z^{-n-1}
\end{equation}
are mutually local with respect to their operator product expansion
\begin{equation}
\varphi(z)\psi(\omega)\sim\frac{(\varphi|\psi)K}{(z-\omega)}.
\end{equation}

\N{Example}
Let $A$ be the odd 1-dimensional superspace $\CC\varphi$ with bilinear
form $(\varphi|\varphi)=1$. Let $K=1$. THen $C_{A}$ turns into the
algebra
\begin{equation}
\varphi_{m}\varphi_{n}+\varphi_{n}\varphi_{m}=\delta_{m,-n}
\end{equation}
for $m,n\in\frac{1}{2}+\ZZ$. The (odd) formal distribution
\begin{equation}
\varphi(z)=\sum_{m\in\ZZ}\varphi_{m+1/2}z^{-m-1}
\end{equation}
is called a \define{Free Neutral Fermion}. Its operator product
expansion is
\begin{equation}
\varphi(z)\varphi(\omega)\sim\frac{1}{z-\omega}.
\end{equation}

\N{Example}
Let $A$ be the odd 2-dimensional superspace
$\CC\varphi^{+}\oplus\CC\varphi^{-}$ with the supersymmetric bilinear
form
\begin{equation}
(\varphi^{+}|\varphi^{-})=0,\quad\mbox{and}\quad
(\varphi^{\pm}|\varphi^{\pm})=1.
\end{equation}
Again let $K=1$. Then we obtain the algebra
\begin{equation}
\varphi^{\pm}_{m}\varphi^{\mp}_{n}+\varphi^{\mp}_{n}\varphi^{\pm}_{m}=\delta_{m,-n}
\end{equation}
and
\begin{equation}
\varphi^{\pm}_{m}\varphi^{\pm}_{n}+\varphi^{\pm}_{n}\varphi^{\pm}_{m}=0.
\end{equation}
The odd formal distributions
\begin{equation}
\varphi^{\pm}(z)=\sum_{n\in\ZZ}\varphi^{\pm}_{n+1/2}z^{-n-1}
\end{equation}
are called \define{Charged Free Fermions} and their operator product
expansion is
\begin{equation}
\varphi^{\pm}(z)\varphi^{\mp}(\omega)\sim\frac{1}{z-\omega}
\end{equation}
and
\begin{equation}
\varphi^{\pm}(z)\varphi^{\pm}(\omega)\sim0.
\end{equation}
