\N{Definition} The set of all possible outcomes of an experiment is
called the \define{Sample Space} and denoted
$\sampleSpace$. An \define{Elementary Event} is an element of
$\sampleSpace$, whereas a \define{Event} is a subset of $\sampleSpace$.

\textsc{Caution:} We will abuse notation, and mix up the singleton
$\{x\}$ with the element $x$. So $\{x\}$ is an elementary event, and
usually just referred to as $x$.

\N{Example} Flipping a coin has two results: heads $H$, tails $T$. The
sample space is
\begin{equation}
\sampleSpace = \{\,H,T\,\}.
\end{equation}
What's an event? Well, lets consider a few:
\begin{enumerate}
\item We flip the coin and get a heads.
\item We get either a heads or a tails.
\item The outcome is both a heads and tails.
\item The outcome is not a heads.
\end{enumerate}
Note that the first and last examples are elementary events, the others
are not elementary.

\N{Remark}
This process ``flipping a coin'', is generalized in mathematics to any
experiment with two outcomes: either heads or tails; the baby is either
a boy or a girl; the cat is either dead or alive\footnote{When
  observed!}. This experiment is called a \define{Bernoulli trial}, and
it's the foundation of most (all?) of probability theory.

\N{Example}
Not all sample spaces are finite. For example, consider an experiment
describing the decay of an unstable particle. How long does it take?
Well, the sample space would be
\begin{equation}
\sampleSpace = \{x\in\RR : x\geq0\}.
\end{equation}
This is quite infinite!

\N{Definition}
We want to think of subsets of the sample space as \emph{events}. The
sample space is a ``certain event'': something's \emph{definitely} going
to happen. So now we want to define the ``collection of all events (of
our sample space)''\dots{}but not every subset is an event! So we need
some axioms/specifications.

We define a \define{$\sigma$-Field} (or \emph{$\sigma$-Algebra})
$\mathcal{F}$ to be the set of events of our sample space
$\sampleSpace$. But that's not the end of the  story: we have a bunch of
axioms to consider. 

First, it seems sound to suggest for any pair of events $A$ and $B$
(i.e., $A,B\in\mathcal{F}$), we
can form new events ``$A$ and $B$'' as well as ``$A$ or $B$''. These
correspond to the operations
\begin{equation*}
\mbox{$A$ and $B$}=A\cap B,\quad\mbox{and}\quad
\mbox{$A$ or $B$}=A\cup B.
\end{equation*}
Good, well, so what?

\begin{axiom}[Closed under pair-wise ``And'', ``Or''] 
If $A,B\in\mathcal{F}$, then $A\cup B\in\mathcal{F}$ and $A\cap
B\in\mathcal{F}$. 
\end{axiom}

Under a similar vein of reasoning, if we have an event
$A\in\mathcal{F}$, then its complement $\comp{A}$ (read ``The event that
$A$ does not occur'') should also be an event:
$\comp{A}\in\mathcal{F}$. So, we have
\begin{axiom}[Closed under complements]
If $A\in\mathcal{F}$, then $\comp{A}\in\mathcal{F}$.
\end{axiom}

The last axiom is quite simple: nothing is an event. What's ``nothing''?
The empty set:
\begin{axiom}[Nothing is an event]
We have $\emptyset\in\mathcal{F}$.
\end{axiom}
Is this really the last axiom? No, we weren't honest with our first
axiom. We have something \emph{more}: we could have an infinite number
of ``and'' (but not an infinite number of ``or'').
\begin{axiom}
If $A_{i}\in\mathcal{F}$, then $\displaystyle\bigcup_{i=1}^{\infty}A_{i}\in\mathcal{F}$
\end{axiom}
When is this useful? Suppose we want to flip a coin, and keep flipping
until we get a heads. What's the sample space look like? Well, it'd be
\begin{equation}
\sampleSpace=\{\, H,TH, TTH, TTTH, \dots\,\}.
\end{equation}
The event that we flip the coin an even number of times is
\begin{equation}
E = \{\, TH, TTTH, TTTTTH, \dots\,\}.
\end{equation}
Unless we have this last axiom, we couldn't construct it!

\N{Example} The smallest $\sigma$-algebra associated to any sample space
$\sampleSpace$ is 
\begin{equation}
\mathcal{F}=\{\emptyset,\sampleSpace\}.
\end{equation}
It ``obviously'' satisfies the four axioms.

\N{Example} The next smallest algebra associated to $\sampleSpace$ is,
if $A$ is any subset of $\sampleSpace$, then
\begin{equation}
\mathcal{F}=\{\,\emptyset,A,\comp{A},\sampleSpace\,\}.
\end{equation}
Although a little trickier to show, it also satisfies the axioms.

\N{Example} 
When $\sampleSpace$ is finite\footnote{For \emph{infinite}
sample spaces, things get tricky because we're really going to do
``integration'' on our set. For the real numbers, for example, its
powerset includes the natural numbers\dots but an integral over the
natural numbers embedded in the reals is zero! We get strange results 
like that: where things should have some probability, they instead
have none.}, its powerset (the set of all subsets) of $\sampleSpace$ is
a $\sigma$-algebra. This is the most common $\sigma$-algebra used when
working with finite sample spaces. 
