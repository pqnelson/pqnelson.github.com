\N{Definition}
Let $x\in\RR$ be any real number. We define $\intPart{x}$ to be the
greatest integer smaller than $x$. So observe
\begin{subequations}
\begin{align}
\intPart{3.1} &= 3\\
\intPart{\pi} &= 3\\
\intPart{-\E} &=-3
\end{align}
\end{subequations}
We always will have
\begin{equation}
\intPart{x}\leq x.
\end{equation}
We now may define
\begin{equation}
\fracPart{x}=x-\intPart{x}=\mbox{fractional part of $x$}
\end{equation}
and observe $0\leq\fracPart{x}<1$.

\begin{lemma}
Suppose we have a sequence $\{c_{n}\}^{\infty}_{n=1}$ and for $x\geq1$
we define a function
\begin{equation}
C(x) = \sum_{n\leq x}c_n,\quad\mbox{and}\quad C(0)=0.
\end{equation}
Let $f$ be a $C^{1}$ function, then
\begin{equation}
\sum_{n\leq x}c_{n}f(n) = C(x)f(x)-\int^{x}_{1}C(t)f'(t)\,\D t.
\end{equation}
\end{lemma}
\begin{proof}
This is a two-step proof. Step one notes
\begin{subequations}
\begin{equation}
\sum_{n\leq x}C(n)\bigl(f(n+1)-f(n)\bigr) =
C(x)f(\intPart{x})-\sum_{n\leq x}c_{n}f(n)
\end{equation}
but the left hand side is precisely
\begin{equation}
\sum_{n\leq x}C(n)\bigl(f(n+1)-f(n)\bigr) =
\int^{\intPart{x}}_{1}C(t)f'(t)\,\D t.
\end{equation}
Thus we obtain
\begin{equation}\label{eq:pf:partialSum:stepOne}
\int^{\intPart{x}}_{1}C(t)f'(t)\,\D t=C(x)f(\intPart{x})-\sum_{n\leq x}c_{n}f(n).
\end{equation}
\end{subequations}
That concludes the first step.

The second step notes
\begin{equation}\label{eq:pf:partialSum:stepTwo}
\int^{x}_{\intPart{x}}C(t)f'(t)\,\D t=C(x)f(x)-C(x)f(\intPart{x}).
\end{equation}
This uncontroversial statement should be seen immediately by the
fundamental theorem of calculus.

We then add Eq \eqref{eq:pf:partialSum:stepOne} to Eq \eqref{eq:pf:partialSum:stepTwo}
to find
\begin{subequations}
\begin{multline}
\int^{x}_{\intPart{x}}C(t)f'(t)\,\D t
+\int^{\intPart{x}}_{1}C(t)f'(t)\,\D t\\
=C(x)f(x)-C(x)f(\intPart{x})+C(x)f(\intPart{x})-\sum_{n\leq x}c_{n}f(n)
\end{multline}
which simplifies to
\begin{equation}
\int^{x}_{1}C(t)f'(t)\,\D t=C(x)f(x)+\sum_{n\leq x}c_{n}f(n)
\end{equation}
\end{subequations}
precisely as desired.
\end{proof}


\N{Example (Harmonic Series)}
We will apply our lemma to the harmonic series. How? Well, consider
\begin{equation}
H_{x}=\sum_{n\leq x}\frac{1}{n}
\end{equation}
which we consider the sequence $c_{n}=1$ and $f(t)=1/t$. Then note
\begin{equation}
C(x)=\sum_{n\leq x}1=\intPart{x}.
\end{equation}
Thus our lemma implies
\begin{equation}
H_{x} = \frac{\intPart{x}}{x}+\int^{x}_{1}\frac{\intPart{t}}{t^{2}}\,\D
t.
\end{equation}
We will try to simplify this.

First we should note that $x=\intPart{x}+\fracPart{x}$. Thus
\begin{subequations}
\begin{align}
H_{x}&=\frac{x-\fracPart{x}}{x}+\int^{x}_{1}\frac{t-\fracPart{t}}{t^{2}}\D
t\\
&=\ln(x)+\left(1-\frac{\fracPart{x}}{x}\right)-\int^{x}_{1}\frac{\fracPart{t}}{t^{2}}\D t\\
&=\ln(x)+\left(1-\int^{\infty}_{1}\frac{\fracPart{t}}{t^{2}}\D t\right)
+\int^{\infty}_{x}\frac{\fracPart{t}}{t^{2}}\D t-\frac{\fracPart{x}}{x}
\end{align}
\end{subequations}
Let
\begin{equation}
\gamma=1-\int^{\infty}_{1}\frac{\fracPart{t}}{t^{2}}\D t
\end{equation}
then our expression simplifies to
\begin{equation}
H_{x} = \ln(x)+\gamma+\int^{\infty}_{x}\frac{\fracPart{t}}{t^{2}}\D
t-\frac{\fracPart{x}}{x}.
\end{equation}
Wonderful.

But does the integral term converge? We see
\begin{subequations}
\begin{align}
\left|\int^{\infty}_{x}\frac{\fracPart{t}}{t^{2}}\D t\right|
&\leq\int^{\infty}_{x}\frac{|\fracPart{t}|}{t^{2}}\D t\\
&\leq\int^{\infty}_{x}\frac{1}{t^{2}}\D t = \frac{1}{x}.
\end{align}
\end{subequations}
Moreover this implies
\begin{equation}
\int^{\infty}_{x}\frac{\fracPart{t}}{t^{2}}\D
t-\frac{\fracPart{x}}{x}=\bigO(x^{-1}).
\end{equation}
So the expression for $H_{x}$ becomes
\begin{equation}
H_{x}=\ln(x)+\gamma+\bigO(x^{-1})
\end{equation}
but \emph{can we do better?}

\M
Lets try to figure out the next terms to order $\bigO(x^{-2})$. We will
consider
\begin{equation}
\int^{\infty}_{x}\frac{\fracPart{t}}{t^{2}}\D t-\frac{\fracPart{x}}{x}
=\int^{\infty}_{x}\frac{\fracPart{t}-\frac{1}{2}}{t^{2}}\D
t+\frac{1}{2}\int^{\infty}_{x}\frac{\D t}{t^{2}}-\frac{\fracPart{x}}{x}
\end{equation}
which simplifies to
\begin{equation}
\int^{\infty}_{x}\frac{\fracPart{t}}{t^{2}}\D t-\frac{\fracPart{x}}{x}
=\int^{\infty}_{x}\frac{\fracPart{t}-\frac{1}{2}}{t^{2}}\D t
+\frac{1}{2x}-\frac{\fracPart{x}}{x}.
\end{equation}
Now we make a claim!

\N*{Claim} The integral $\displaystyle{\int^{\infty}_{x}\frac{\fracPart{t}-\frac{1}{2}}{t^{2}}\D t=\bigO(1/x^{2})}$.

\begin{proof}
We will integrate by parts, using
\begin{equation}
u=\frac{1}{t^{2}},\quad\mbox{and}\quad\D u=\frac{-2}{t^{3}}\D t
\end{equation}
and
\begin{equation}
\D v =\left(\frac{1}{2}-\fracPart{t}\right)\D t,\quad\mbox{and}\quad
v = \int^{t}_{1}\left(\frac{1}{2}-\fracPart{u}\right)\D u.
\end{equation}
Thus the integral becomes
\begin{multline}
\lim_{R\to\infty}\int^{R}_{x}\frac{\fracPart{t}-\frac{1}{2}}{t^{2}}\D t\\
=\left.\frac{1}{t^{2}}\int^{t}_{1}\left(\frac{1}{2}-\fracPart{u}\right)\D u\right|^{t=R}_{t=x}
+2\int^{R}_{x}\frac{1}{t^{3}}\int^{t}_{1}\left(\frac{1}{2}-\fracPart{u}\right)\D u
\end{multline}
We claim that
\begin{equation}
\begin{split}
v &= \int^{[t]}_{1}
\left(\frac{1}{2}-\fracPart{u}\right)\D u
+\int^{t}_{[t]}
\left(\frac{1}{2}-\fracPart{u}\right)\D u\\
&=0+\int^{t}_{[t]}
\left(\frac{1}{2}-\fracPart{u}\right)\D u
\end{split}
\end{equation}
How can we see this? Well, we should note for any integer $k$ that
\begin{equation}
\int^{k+1}_{k}\left(\frac{1}{2}-\fracPart{u}\right)\D u=0,
\end{equation}
since it's a sawtooth function. Thus
\begin{equation}
v\leq\int^{1}_{1/2}\left(\frac{1}{2}-\fracPart{u}\right)\D
u=\frac{1}{4}.
\end{equation}
Our integral becomes
\begin{equation}
\begin{split}
\lim_{R\to\infty}\int^{R}_{x}\frac{\fracPart{t}-\frac{1}{2}}{t^{2}}\D t
&=\lim_{R\to\infty}\left.\frac{1}{t^{2}}\frac{1}{4}\right|^{t=R}_{t=x}
+2\int^{R}_{x}\frac{1}{t^{3}}\frac{1}{4}\D t\\
&\sim\bigO(x^{-2}).
\end{split}
\end{equation}
This proves the claim.
\end{proof}
Moreover, for any integer $n$, we see
\begin{equation}
\sum^{N}_{n=1}\frac{1}{n} = \ln(N)+\gamma+\frac{1}{2N}+\bigO(N^{-2}).
\end{equation}
This turns out to be a good \emph{asymptotic} approximation for harmonic
numbers. 
