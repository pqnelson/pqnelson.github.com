\N{Overview}
This section is just to establish the notation used for set theory.
We use a ``naive set theory'', which --- for the author --- is really
just ZF+GC. 

\N{Global Choice}
Recall we have quantifiers $\forall$ and $\exists$. We can also have a
quantifier $\varepsilon$ which has the form
\begin{equation}
\varepsilon x : P(x)
\end{equation}
and it returns the object $x$ which satisfies the predicate $P(x)$, if
one exists. If there is no such $x$ (e.g., $P(x)$ is a contradiction),
then it returns an arbitrary object.

This $\varepsilon$ operator is called the \define{Global Choice
Operator}. 

\N{Definition} A \define{Set} is a well-defined collection of ``stuff''
without duplicates.

\N{Definition}
If $X$ is a set, and $x$ is an object, if $x$ lives in the collection
$X$ we write $x\in X$ and call it a \define{Element} or \define{Member}
of $X$. We will write sets using capital Latin letters unless otherwise
indicates. If $y$ does not belong to $X$, we write $y\notin X$.

Note \emph{any type of object} can belong to a set. For example, we can
have a set of selected sets, the set of integers (usually denoted
$\mathbb{Z}$), etc.

\textsc{Caution}: It is illegal (``meaningless'') to write $X\in X$ or $X\notin X$. 

\N{Example} The empty collection is a set, denoted $\emptyset$ and
called the \define{Empty Set}. It is defined by the condition, for any
$x$, we have $x\notin \emptyset$.

\N{Example} 
The collection of natural numbers $\NN=\{\,1,2,3,\dots\,\}$, the natural
numbers with zero $\NN_{0}=\{\,0,1,2,3,\dots\,\}$. The integers
$\ZZ=\{\dots,-1,0,1,\dots\,\}$. These are all sets.

\N{Non-Example (Universe)}
Consider $\mathcal{U}$ the well-defined collection of all sets. It is
not a set, since it is illegal to write $\mathcal{U}\in\mathcal{U}$. No,
the universe is usually something ``bigger'' than a set (it's
a \emph{class}). We usually don't work with classes, and so we won't
worry about them. But we'd like to note the collection of all sets
$\mathcal{U}$ is called the \define{Universe}.

\N{Definition}
Let $X$ and $Y$ be sets. If every element $x\in X$ belongs to $Y$, and
every $y\in Y$ belongs to $X$, then we say the two sets are
\define{Equal} and write $X=Y$.

\N{Definition}
Let $X$ and $Y$ be sets. If every element $x\in X$ belongs to $Y$, then
we write $X\subset Y$ and call $X$ a \define{Subset} of $Y$.

\N{Theorem} We have $X=Y$ if and only if $X\subset Y$ and $Y\subset X$.

\N{Definition}
Let $Y$ be a set. A \define{Proper Subset} $X$ of $Y$ consists of a
subset that is not equal to $Y$. That is: $X\subset Y$ and $X\neq Y$. We
indicate proper subsets by writing $X\propersubset Y$.

\N{Examples} Observe we have $\NN\propersubset\NN_{0}\propersubset\ZZ$.

\N{Example} For any set $X$, we have $X\subset X$ but
$X\not\propersubset X$.

\N{Definition}
Let $X$ be any set. Then the \define{Power Set} of $X$ is the collection
$\powerset(X)$ of subsets $Y\subset X$. Note this implies
$X\in\powerset(X)$. 

\N{Proposition}
Let $X$ be any set. Then $\emptyset\in\powerset(X)$ and
$X\in\powerset(X)$.

\N{Definition}
An \define{Ordered Tuple} $(a,b)$ is a pair of mathematical objects $a$,
$b$. The first slot $a$ is the first component (sometimes called
the \emph{first coordinate}).

We write $(a,b)=(x,y)$ if and only if $a=x$ and $b=y$.

More generally, if we have $n$ objects, we can form the ordered
$n$-tuple $(x_{1},\dots,x_{n})$. Again, equality is defined
component-wise. 

\N{Definition}
Let $X$ and $Y$ be sets. Then their \define{Cartesian Product} is a set
$X\times Y$ consisting of ordered pairs $(x,y)\in X\times Y$ where $x\in
X$, $y\in Y$.

Again, we can generalize this to the Cartesian product of any number of
sets $X_1\times\dots\times X_n$ consisting of ordered $n$-tuples.

\N{Definition}
A \define{Function} $f\colon X\to Y$ associates to each $x\in X$
precisely one $y\in Y$ usually denoted $y=f(x)$. 

Sometimes mathematicians assert functions are sets. Well, a function $f$
is a subset $f\subset X\times Y$ with the property for each $x\in X$, we
have precisely one ordered pair $(x,y)\in f$.

\N{Example}
Let $X$ be any set.
The identity function $\mathrm{id}\colon X\to X$ defined by
$\mathrm{id}(x)=x$ is a function on $X$.
