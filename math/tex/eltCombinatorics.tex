\M
Probability amounts to calculate
\begin{equation}
\Pr(A)=\frac{\begin{pmatrix}\mbox{number of}\\
\mbox{elements in $A$}
  \end{pmatrix}}{|\sampleSpace|}.
\end{equation}
This is both a blessing and a curse: for we change probability into
counting, yet counting can be difficult! We will review elements of
combinatorics (``the art of counting'').

\subsection{Permutations}
\M
Suppose we want to consider how many ways we can order $n$
elements. What to do?

We first say ``We have $n$ slots to fill with our $n$ elements.'' Then
we pick some element, and note how many possible slots we may choose. We
may pick any one of our $n$ slots. Then pick our second element, and we
may place it in any of the remaining $(n-1)$ slots. We continue, and
there are 
\begin{equation}
n(n-1)(n-2)(\cdots)(1)=n!
\end{equation}
different methods of ordering our $n$ elements. Thus there are $n!$
different permutations.

\N*{Convention:} We set $0!=1$.

\M
Factorial grows quite rapidly. For example
\begin{subequations}
\begin{align}
3!&=6\\
6!&=720\\
12!&=479\,001\,600
\end{align}
\end{subequations}
We may approximate it using \define{Stirling's Approximation}
\begin{equation}
n!\sim\sqrt{2\pi n}\left(\frac{n}{\E}\right)^{n}.
\end{equation}
We may derive this from the Gamma function
\begin{equation}
\Gamma(x)=\int^{\infty}_{0}t^{x-1}\E^{-t}\,\D t,\qquad\mbox{for
}\Re(x)>0.
\end{equation}
We take $\Gamma(n+1)$ and integrate by parts $n$ times, we find
\begin{equation}
\Gamma(n+1)=n!
\end{equation}
for $n\in\NN$.

Integrating $\Gamma(x+1)$ by parts merely once will give us the
functional equation
\begin{equation}
\Gamma(x+1)=x\Gamma(x).
\end{equation}
Calculus students around the world prove everyday that
\begin{equation}
\Gamma(1/2)=\sqrt{\pi}
\end{equation}
without ever realizing it!

\subsection{Sampling}
\M
Given some collection of elements (or a ``population''), we would like
to pick a finite number of elements (a ``sample''). A population with
$n$ elements $\{a_{1},\dots,a_{n}\}$, any \emph{ordered} arrangement of
$r$ symbols $a_{j_{1}}$, \dots, $a_{j_r}$ is an \define{Ordered Sample
  of Size $r$} drawn from the population. Note when $r=n$, an ordered
sample is just a permutation of the population.

\N*{Sampling with Replacement:}
The elements are drawn from the entire population, so we may pick the
same element twice (or more).

\N*{Sampling Without Replacement:}
Once we choose an element, we cannot pick it again. We reduce the
possible choices, and hence more limited than sampling with replacement.

\medbreak
Choosing $r$ elements with replacement from a population of $n$ can be
done in $n^{r}$ different ways. The proof is obvious: you have $r$
choices, and each choice has $n$ possibilities. So there are 
\begin{equation}
\underbrace{n\cdot(\cdots)\cdot n}_{\text{$r$ times}} = n^{r}
\end{equation}
different ways to sample.

How many different ways can we sample $r$ elements without replacement
from a population with $n$ elements? 

\M
Since ordering matters, we have
\begin{equation}
n(n-1)(\cdots)\bigl(n-(r-1)\bigr)=(n)_{r}
\end{equation}
This defines the \define{Falling Factorial} $(n)_{r}$.

How can we be certain about this? Well, for the first choice there are
$n$ possibilities. The second has $(n-1)$ possibilities. The $k^{th}$
has $(n-(k-1))$ possible choices. We then multiply these together to get
the result.

\N{Birthday Problem}
How many people do we need in a room to make a favorable bet (i.e., a
bet with probability of success greater than $1/2$) that 2 people in the
room will have the same birthday?

\N*{Solution:}
Let $r$ be the number of people in the room, we want to find it. We will
compute the probability of no repetition in our data (i.e., the
probability no one in the room shares a birthday). The population, for
this example, is the 365 days of the year. 

Then
\begin{equation}
p_{r} = \frac{(365)_{r}}{365^{r}}
\end{equation}
describes the probability that no one shares a birthday. We want to find
the smallest $r$ such that $p_{r}<0.5$. This can be done by exhaustion,
we find
\begin{equation}
\begin{split}
p_{23} &=
\frac{36997978566217959340182499134166757044383351847256064}{75091883268515350125426207425223147563269805908203125}\\
&\approx 0.49270276
\end{split}
\end{equation}
So only make this bet when there are at least 23 people in the room!

\subsection{Binomial Coefficients}
\M
How many samples of size $r$ exist in a population of $n$ elements,
without regard to ordering? That is, how many ways can we choose $r$
guys from $n$ elements? We count this using binomial coefficients
\begin{equation}
{n \choose r} = \frac{n!}{r!(n-r)!}
\end{equation}
and read it as ``$n$ choose $r$''. Why is this true? Great question: we
have $(n)_r$ subpopulations, and $r!$ different orderings of each
subpopulation, so we have
\begin{equation}
{n\choose r}=\frac{(n)_{r}}{r!}.
\end{equation}
But this is precisely what we have written!

\N{Example}
A deck of cards has 52 cards, a poker hand has 5 cards.
How many different poker hands are there?

\N*{Solution:}
We see there are
\begin{equation}
{52\choose 5}=2\,598\,960
\end{equation}
different poker hands.

\N{Example}
Consider a random distribution of $r$ balls in $n$ urns. Find the
probability $p_{k}$ that a specified cell contains exactly $k$ balls.

\N*{Solution:}
We want to find
\begin{equation}
p_{k} = \frac{\begin{pmatrix}
    \mbox{Number of ways to}\\
    \mbox{choose $k$ balls from}\\
    \mbox{$r$ balls}
  \end{pmatrix}
\begin{pmatrix}
  \mbox{number of ways to}\\
  \mbox{put remaining $(r-k)$ balls}\\
  \mbox{in $(n-1)$ cells}
\end{pmatrix}}{\begin{pmatrix}\mbox{number of different}\\
\mbox{ways placing $r$}\\
\mbox{balls in $n$ cells}
  \end{pmatrix}}
\end{equation}
There are $n^{r}$ different ways placing $r$ balls in $n$ cells. This
gives us the denominator. We see that
\begin{equation}
\begin{pmatrix}
    \mbox{Number of ways to}\\
    \mbox{choose $k$ balls from}\\
    \mbox{$r$ balls}
  \end{pmatrix} = {r\choose k}
\end{equation}
and what of the remaining $(r-k)$ balls? They can be placed in the
remaining cells in $(n-1)^{r-k}$ ways. Thus we plug this all back into
our equation to find
\begin{equation}
\begin{split}
p_{k}&=\frac{{r\choose k}(n-1)^{r-k}}{n^{r}}\\
&={r\choose k}\frac{1}{n^{k}}\left(1-\frac{1}{n}\right)^{r-k}.
\end{split}
\end{equation}
This is an example of the famous \emph{binomial distribution}, which
we'll soon study.

\N{Example}
The US Senate has 100 Senators (a pair represents each state). If a
committee is formed with 50 senators, then:

(a) What's the probability a given state is represented?

(b) What's the probability each state is represented?

\N*{Solution}
(a) Well, we first consider the \emph{negation} of this statement: what's
the probability not every state is represented? There are
\begin{equation}
{98\choose 50} = d_{k}
\end{equation}
ways to choose the committee so some state is not represented. But there
is 
\begin{equation}
{100\choose 50}=n_{k}
\end{equation}
different ways to choose the committee. Thus the probability a given state is
not represented is
\begin{equation}
q = \frac{{98\choose50}}{{100\choose50}} = \frac{49}{198}\approx 0.24747\dots
\end{equation}
and hence the probability a given state is represented is
\begin{equation}
1-q = \frac{149}{198} \approx 0.7525\dots
\end{equation}
That concludes the first question.

(b)
We pick one senator from each state. We have two possibilities for each
state, and 50 states, hence we have $2^{50}$ possibilities. Thus we have
\begin{equation}
p = \frac{2^{50}}{{100\choose 50}} \approx 1.11\times10^{-14}
\end{equation}
is the probability every state is represented.
