\M
We have introduced rudiments of probability theory, but lets consider
situations when we use randomness in life. Consider this motivating
example: flip a coin twice. The sample space is
\begin{equation}
\sampleSpace=\{\,HH,HT,TH,TT\,\}.
\end{equation}
You bet \$1. If you get $HH$, you get \$2 back; otherwise you lose. How
do we describe this game?

Why not introduce a function $W\colon\sampleSpace\to\RR$ defined by
\begin{equation}
W(HH)=2,\quad\mbox{and}\quad W(HT)=W(TH)=W(TT)=0.
\end{equation}
Can we do this? Why not! But what is this function? We call it a
\define{Random Variable}.

\N{Definition}
Let $(\sampleSpace,\mathcal{F},\Pr)$ be a probability space. We define a
\define{Random Variable} to be a function $W\colon\sampleSpace\to\RR$
such that for each $x\in\RR$ we have the set
\begin{equation}
X = \{ \omega\in\sampleSpace : W(\omega)\leq x\}
\end{equation}
be an element of $\mathcal{F}$, or in symbols $X\in\mathcal{F}$. 

\begin{rmk}
We will consider the simpler case of \define{Discrete Random Variables}
$X\colon\sampleSpace\to\ZZ$. 
\end{rmk}

\N{Example/Definition}
Let $X$ be a random variable on $(\sampleSpace,\mathcal{F},\Pr)$. The
\define{Distribution Function} of $X$ is the function
\begin{equation}
F\colon\RR\to[0,1]
\end{equation}
defined by
\begin{equation}
F(x)=\Pr(X\leq x)
\end{equation}
for $x\in\RR$.

Note we use an abbreviation
\begin{equation}
\Pr(X\leq x)=\Pr(\{\omega\in\sampleSpace : X(\omega)\leq x\}).
\end{equation}
This is a subtle point, but an important one!

\N{Example}
Suppose we consider the situation flipping a coin twice, with the random
variable
\begin{equation}
W(HH)=2,\quad\mbox{and}\quad W(HT)=W(TH)=W(TT)=0.
\end{equation}
What is the distribution of $W$?

\N*{Solution:}
We see that
\begin{subequations}
\begin{equation}
\Pr(W\leq 1)=\frac{3}{4}
\end{equation}
and
\begin{equation}
\Pr(W\leq2)=1.
\end{equation}
\end{subequations}
The latter makes intuitive sense since $W(\sampleSpace)=\{0,2\}$. 

\N{Lemma}
A distribution function $F$ for the random variable $X$ has the
following properties: 
\begin{enumerate}
\item $\displaystyle\lim_{x\to-\infty}F(x)=0$
\item $\displaystyle\lim_{x\to+\infty}F(x)=1$
\item If $x<y$, then $F(x)\leq F(y)$
\item We have $F$ be right-continuous, i.e., $F(x+h)\to F(x)$ as
  $h\downarrow 0$.
\end{enumerate}

\begin{proof}
We will prove each proposition.
\begin{enumerate}
\item Consider the sequence of sets $B_{n}=\{\omega\in\sampleSpace :
  X(\omega)\leq-n\}$. We see then that $\dots\subset B_{n+1}\subset
  B_{n}\subset\dots\subset B_{0}$ and obviously $\emptyset$ is the limit
  of this sequence. We have
\begin{equation}
B=\bigcap^{\infty}_{n=0}B_{n}
\end{equation}
satisfy 
\begin{equation}
\Pr(B)=\lim_{n\to\infty}\Pr(B_{n}).
\end{equation}
Then we see $\Pr(B)=\Pr(\emptyset)=0$.
\item The proof is similar, we just have $B_{0}\subset
  B_{1}\subset\dots$ which has its limit be $\sampleSpace$. Thus
  $\Pr(B)=\Pr(\sampleSpace)=1$. 
\item We see that
\begin{subequations}
\begin{equation}
F(x) = \Pr(A)
\end{equation}
where
\begin{equation}
A=\{\omega\in\sampleSpace : X(\omega)\leq x\}
\end{equation}
and similarly
\begin{equation}
F(y)=\Pr(B)
\end{equation}
where
\begin{equation}
A=\{\omega\in\sampleSpace : X(\omega)\leq y\}.
\end{equation}
We see, since $x<y$, that $A\subset B$. Moreover this implies
\begin{equation}
\Pr(A)\leq\Pr(B)
\end{equation}
\end{subequations}
which proves the statement.
\item Consider the sequence of sets
\begin{subequations}
\begin{equation}
B(h) = \{\omega\in\sampleSpace : X(\omega)\leq x+h\}
\end{equation}
We want to prove as $h\downarrow0$ that
$\Pr\bigl(B(h)\bigr)\to\Pr\bigl(B(0)\bigr)$. We see that, if we write 
\begin{equation}
\Delta(h)=B(h)\setminus B(0)
\end{equation}
then we have $B(0)$ and $\Delta(h)$ be disjoint. Thus
\begin{equation}
\Pr\bigl(B(0)\cup\Delta(h)\bigr)-\Pr\bigl(B(0)\bigr)=\Pr\bigl(\Delta(h)\bigr)
\end{equation}
by lemma \ref{lemma:probSpaceProps}. But look, for any sequence
$h_{n}\to0$ as $n\to\infty$, write $\Delta_{n}=\Delta(h_{n})$, then we
have a strictly decreasing sequence
\begin{equation}
\dots\subset\Delta_{n+1}\subset\Delta_{n}\subset\dots\subset\Delta_{0}.
\end{equation}
This has its limit be $\emptyset$, and again this implies
\begin{equation}
\lim_{n\to\infty}\Pr(\Delta_{n})=0
\end{equation}
which proves right-continuity. \qedhere
\end{subequations}
\end{enumerate}
\end{proof}
